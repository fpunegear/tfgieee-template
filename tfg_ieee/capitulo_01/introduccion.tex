\fancyhead{}
\fancyfoot{}
\lhead{Introducción}
\cfoot{\thepage}

\chapter{Introducción}

El registro de asistencia es una tarea clave en la gestión educativa y administrativa en instituciones académicas como la Facultad Politécnica de la Universidad Nacional del Este (FPUNE), desempeñando un rol fundamental en el seguimiento del desempeño estudiantil. Sin embargo, los métodos tradicionales de control manual mediante listas de asistencia suelen presentar errores humanos que afectan la precisión de los registros y la toma de decisiones académicas.

Este Trabajo Final de Grado (TFG) tiene como objetivo desarrollar un sistema automatizado basado en reconocimiento facial para optimizar el proceso de registro de asistencia en la FPUNE. Para ello, se utilizarán algoritmos avanzados de biometría facial que analizarán imágenes capturadas en tiempo real, asegurando así una identificación precisa y eficiente. Además, se ofrecerá una plataforma digital donde los estudiantes podrán justificar sus ausencias, y las autoridades correspondientes podrán gestionar estas justificaciones.

La metodología empleada comprende etapas de selección tecnológica, recopilación y procesamiento de imágenes, entrenamiento de modelos inteligentes y pruebas piloto en un entorno real durante un semestre académico del año 2025. Se espera que este sistema reduzca significativamente los errores actuales y mejore la gestión académica y administrativa, brindando datos más fiables y contribuyendo a una mayor eficiencia operativa.

Este estudio busca aportar información valiosa para estudiantes, docentes y autoridades, promoviendo una gestión académica más precisa y efectiva mediante la adopción de tecnologías innovadoras.
\begin{table}[]
\begin{tabular}{lllll}
\end{tabular}
\end{table}
\section{Motivación}
En la actualidad, el control de asistencia en instituciones educativas sigue dependiendo de métodos manuales, como el llamado de lista o el registro en papel, lo que puede generar errores humanos y pérdida de tiempo en la gestión académica. La automatización de este procedimiento mediante tecnologías de reconocimiento facial se presenta como una alternativa innovadora que no solo mejora la precisión del registro de asistencia, sino que también permite una gestión más eficiente de la información en tiempo real.

En el ámbito académico, este estudio representa una oportunidad para explorar el potencial del reconocimiento facial en entornos educativos, contribuyendo al desarrollo de nuevas aplicaciones basadas en inteligencia artificial y visión computacional. Dado que esta tecnología ha sido implementada con éxito en otros sectores, su adaptación al contexto universitario puede marcar una diferencia significativa en la optimización de procesos.

Este proyecto también busca reducir el impacto ambiental al eliminar el uso de papel en los registros de asistencia, promoviendo prácticas más sostenibles dentro de la institución. En este sentido, la motivación principal radica en la combinación de innovación tecnológica, eficiencia administrativa y sostenibilidad, factores clave que justifican la importancia de esta investigación.
\section{Definición del problema}
En la Facultad Politécnica de la Universidad Nacional del Este (FPUNE), el control de asistencia aún utiliza métodos tradicionales, como el llamado de lista o el registro manual en papel. Estos métodos generan frecuentemente errores humanos, pérdida o deterioro de documentos y dificultades en la organización y almacenamiento eficiente de los datos recopilados. Además, estas fallas afectan negativamente la capacidad de las autoridades académicas y administrativas para detectar patrones claros de asistencia, identificar ausencias repetitivas y aplicar medidas correctivas oportunas.

Ante esta problemática, surge la necesidad de implementar una solución tecnológica basada en reconocimiento facial que permita registrar automáticamente la asistencia. Además, se plantea la creación de una plataforma donde los estudiantes puedan justificar ausencias y entregar los documentos requeridos. En esta plataforma, las personas autorizadas evaluarán y determinarán si las justificaciones son válidas.
\section{Preguntas de Investigación}
\subsection{Pregunta General.}
¿Cómo implementar un sistema de Reconocimiento Facial para el control de asistencia de los estudiantes de la FPUNE?

\subsection{Preguntas Específicas.}
\begin{enumerate}
    \item ¿Cuáles son los algoritmos de reconocimiento facial?
    \item ¿Cuáles son las herramientas tecnológicas para el desarrollo del Sistema?
    \item ¿Cuáles son los requerimientos para el registro de asistencia de la Facultad Politécnica?
    \item ¿Cómo se desarrollará el sistema de reconocimiento facial y gestión de asistencias?
    \item ¿Qué pruebas de implementación se deben realizar para validar su correcto funcionamiento?
    \item ¿Cómo se evaluarán los resultados obtenidos para medir la efectividad del sistema?
    \item ¿Qué indicadores o métricas se pueden utilizar para evaluar el desempeño del sistema implementado?
\end{enumerate}



\section{Objetivo General}
Implementar un sistema de reconocimiento facial para el control de asistencia de estudiantes de la Facultad Politécnica de la Universidad Nacional del Este.

\section{Objetivos Específicos}
\begin{enumerate}
    \item Seleccionar los algoritmos de reconocimiento facial.
    \item Seleccionar las herramientas tecnológicas para el desarrollo del Sistema.
    \item Identificar los requisitos para el desarrollo del sistema de registro de asistencia para la FPUNE.
    \item Desarrollar el sistema de reconocimiento facial y gestión de asistencias.
    \item Realizar pruebas de implementación.
    \item Evaluar resultados obtenidos.
\end{enumerate}




\section{Descripción de los contenidos por capítulo}
Usualmente, el capítulo termina anunciando brevemente el contenido de los restantes capítulos.

